\documentclass[a4paper, 12pt]{article}

\usepackage{graphicx}
\parindent0em

\begin{document}
\begin{center}
\LARGE{\bf BAYESREG OBJECT bayesregObject: regression procedure}
\end{center} 
\vspace{1cm}

\noindent {\bf \large Response:}
\begin{tabbing}
Number of observations: \= 300\\
Response Variable: \> y\\
Family: \> Gaussian\\
\end{tabbing}

\noindent {\bf \large Predictor:}\\


\begin{tabular}{ccp{12cm}}
$\eta$ & $=$ & $\gamma_{const}const + \gamma_{x3}x3 + \gamma_{x4}x4 + f_{x2}(x2) + f_{x1}(x1) + f_{district}(district)$
\end{tabular}
\\ 
\\

\noindent {\bf \large Priors:}\\
\\
Fixed effects:\\
diffuse priors\\
\\\\
$f_{x2}(x2)$\\
first order random walk\\
Inverse gamma prior for variance component with hyperparameters a=0.001 and b=0.001\\
\\\\
$f_{x1}(x1)$:\\
P-spline with second order random walk penalty\\
Number of knots: 20\\
Knot choice: equidistant\\
Degree of Splines: 3\\
Inverse gamma prior for variance component with hyperparameters a=0.001 and b=0.001\\
\\\\
$f_{district}(district)$\\
Markov random field\\
Inverse gamma prior for variance component with hyperparameters a=0.001 and b=0.001\\
\\\\

\noindent {\bf \large MCMC Options:}
\begin{tabbing}
Levels for credible intervals: \= \\
Level 1: \> 95\\
Level 2: \> 80\\
Number of Iterations: \> 10000\\
Burn in: \> 1000\\
Thinning Parameter: \> 10
\end{tabbing}
\vspace{0.5cm}

 {\bf \large Estimation results for the deviance: }\\ 

{\bf Unstandardized deviance } 

\vspace{-0.4cm}
\begin{tabbing}
\hspace{3cm} \= \\
  Mean:           \> 852.43761 \\
  Std. Dev:       \> 11.672088 \\
  2.5\% Quantile:  \> 830.5647 \\
  10\% Quantile:  \> 838.17916 \\
  50\% Quantile:  \> 851.73653 \\
  90\% Quantile:  \> 868.14604 \\
  97.5\% Quantile:  \> 877.28426 \\
\end{tabbing}

{\bf Saturated deviance } 

\vspace{-0.4cm}
\begin{tabbing}
\hspace{3cm} \= \\
  Mean:           \> 298.9705 \\
  Std. Dev:       \> 24.014304 \\
  2.5\% Quantile:  \> 251.86875 \\
  10\% Quantile:  \> 267.60752 \\
  50\% Quantile:  \> 299.39717 \\
  90\% Quantile:  \> 329.67489 \\
  97.5\% Quantile:  \> 345.77093 \\
\end{tabbing}


 {\bf \large Estimation results for the DIC: }\\ 

{\bf DIC based on the unstandardized deviance } 

\vspace{-0.4cm}
\begin{tabbing}
\hspace{3cm} \= \\
deviance($\bar{\mu}$) \> 809.22168 \\
pD  \> 43.215938 \\
DIC  \> 895.65355 \\
\end{tabbing}

{\bf DIC based on the saturated deviance } 

\vspace{-0.4cm}
\begin{tabbing}
\hspace{3cm} \= \\
deviance($\bar{\mu}$) \> 254.50275 \\
pD \> 44.467745 \\
DIC \> 343.43824 \\
\end{tabbing}


 {\bf \large Estimation results for the scale parameter: }\\ 

\vspace{-0.4cm}
\begin{tabbing}
\hspace{3cm} \= \\
Mean  \> 1.01125 \\
Std. dev.:  \> 0.0928975 \\
  2.5\% Quantile:  \>0.851668 \\
  10\% Quantile:  \>0.901505 \\
  50\% Quantile:  \>1.00383 \\
  90\% Quantile:  \>1.13521 \\
  97.5\% Quantile:  \>1.21288 \\
\end{tabbing}


\newpage 


\noindent {\bf \large Fixed Effects:}\\
\\
\begin{tabular}{|r|rrrrr|}
\hline
Variable & Mean & STD & 2.5\%-Quant. & Median & 97.5\%-Quant.\\
\hline
const & -0.0581932 & 0.0850521 & -0.239331 & -0.0573167 & 0.109595\\
x3 & 9.12927 & 0.0663724 & 8.99702 & 9.12844 & 9.26354\\
x4 & 5.2045 & 0.0554732 & 5.09313 & 5.20607 & 5.31853\\
\hline 
\end{tabular}

\newpage
\noindent {\bf \large Plots:}

\begin{figure}[h!]
\centering
\includegraphics[scale=0.6]{res_f_x2_rw.ps}
\caption{Non--linear Effect of 'x2'.
Shown are the posterior means together with 95\% and 80\% pointwise credible intervals.}
\end{figure}

\begin{figure}[h!]
\centering
\includegraphics[scale=0.6]{res_f_x1_pspline.ps}
\caption{Non--linear Effect of 'x1'.
Shown are the posterior means together with 95\% and 80\% pointwise credible intervals.}
\end{figure}

\begin{figure}[h!]
\centering
\includegraphics[scale=0.6]{res_f_district_spatial_pmean.ps}
\caption{Non--linear Effect of 'district'. Shown are the posterior means.}
\end{figure}

\begin{figure}[htb]
\centering
\includegraphics[scale=0.6]{res_f_district_spatial_pcat95.ps}
\caption{Non--linear Effect of 'district'. Posterior probabilities for a nominal level of 95\%.
Black denotes regions with strictly negative credible intervals,
white denotes regions with strictly positive credible intervals.}
\end{figure}

\begin{figure}[htb]
\centering
\includegraphics[scale=0.6]{res_f_district_spatial_pcat80.ps}
\caption{Non--linear Effect of 'district'. Posterior probabilities for a nominal level of 80\%.
Black denotes regions with strictly negative credible intervals,
white denotes regions with strictly positive credible intervals.}
\end{figure}
\end{document}
